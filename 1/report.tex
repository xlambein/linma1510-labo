\documentclass[a4paper,11pt]{article}

\usepackage[in]{fullpage}
\usepackage[utf8]{inputenc}
\usepackage[T1]{fontenc}
\usepackage{textcomp}
\usepackage[english]{babel}
\usepackage{csquotes}

\usepackage{tikz}
\usepackage{graphicx,circuitikz,pgfplots}
\usepackage{tabu,booktabs}
\usepackage{enumitem}
\usepackage{amsmath,amsthm,amsfonts}
\usepackage[squaren,thinspace]{SIunits}
\usepackage[version=3]{mhchem}
\usepackage{bookmark}
\bookmarksetup{numbered,open}
\usepackage[official]{eurosym}
\usepackage{listings}
\usepackage{float}

\newcommand{\equil}[1]{\bar{#1}}
\renewcommand{\d}{\mathrm{d}}

\author{Geoffroy Jacquet \and Xavier Lambein}
\title{Linear Control\\Laboratory report 1}

\begin{document}

\maketitle

\paragraph{Question 1} Explain the chosen values for $\equil{q_{P3}}$ and $\equil{h_3}$. Explain the calculation of the surface $S_{S30}$.

We define $\equil{q_{P3}}$ to be the (constant) flow rate chosen for the open loop experiment (i.e. 30\milli\liter\per\second) and $\equil{h_3}$ to be the value of $h_3$ when the water has stabilized in the third tank. This corresponds to an equilibrium for $h_3$.

In a state of equilibrium, $h_3$ is constant and thus:
\[
    \frac{\d\equil{h_3}}{\d{t}} = 0
    \text.
\]
From that last equation and the model of the system, we can derive the value of $S_{S30}$:
\begin{align*}
    \frac{\d\equil{h_3}}{\d{t}} = 0 = \frac{1}{S_R} \equil{q_{P3}} - \frac{1}{S_R} \equil{q_{S30}}
    &\Leftrightarrow \frac{1}{S_R} \equil{q_{P3}} = \frac{1}{S_R} \equil{q_{S30}} \\
    &\Leftrightarrow \equil{q_{P3}} = \equil{q_{S30}} \\
    &\Leftrightarrow \equil{q_{P3}} = S_{S30} \sqrt{2g\equil{h_3}} \\
    &\Leftrightarrow S_{S30} = \frac{\equil{q_{P3}}}{\sqrt{2g\equil{h_3}}}
\end{align*}

This gives us the following experimental values:
\begin{center}
\begin{tabular}{ccc}
    $\equil{q_{P3}}$ & $\equil{h_3}$ & $S_{S30}$ \\
    \hline
    & &
\end{tabular}
\end{center}

\paragraph{Question 2} Detail the linearized model of the system and the transfer functions of $G(s)$ and $H(s)$.

Let us first identify the input, disturbance, inner state and output of the system:
\begin{center}
\begin{tabular}{lcccc}
    & \textbf{Input} & \textbf{Disturbance} & \textbf{Inner state} & \textbf{Output} \\
    \hline
    \textbf{Variable} & $q_{P3}$ & $S_{F30}$ & $h_3$ & $h_3$ \\
    \hline
    \textbf{Alias} & $u$ & $v$ & $x$ & $y$
\end{tabular}
\end{center}
For clarity, we shall use the ``Alias'' notation for these variables. The (nonlinear) system for this experiment is thus:
\begin{align*}
    \dot{x} &= \frac{S_{S30}}{S_R} \sqrt{2gx} + \frac{1}{S_R} u - \frac{1}{S_R} v \sqrt{2gx} \\
    y &= x
    \text.
\end{align*}

We can obtain a linear system by doing an approximation using the first order Taylor series centered around the equilibrium on the right-hand side of the first equation. Let us begin by computing the gradiant of $\dot{x}$ with respect to $(x, u, v)$:
\[
    \nabla_{(x, u, v)}\dot{x} = \left(
        \frac{S_{S30} - v}{S_R} \sqrt{\frac{g}{2x}},
        \frac{1}{S_R},
        - \frac{\sqrt{2gx}}{S_R}
    \right)
    \text.
\]
The linearization of the system is then given by the Taylor series:
\begin{align*}
    \dot{x}
    &\approx \nabla_{(x, u, v)}\dot{x}\rvert_{(x,u,v) = (\equil{x},\equil{u},\equil{v})} \bullet (x -\equil{x},u - \equil{u},v - \equil{v}) \\
    &= \frac{S_{S30} - \equil{v}}{S_R} \sqrt{\frac{g}{2\equil{x}}} (x - \equil{x})
     + \frac{1}{S_R} (u - \equil{u})
     - \frac{\sqrt{2g\equil{x}}}{S_R} (v - \equil{v}) \\
    &= \frac{S_{S30} - \equil{v}}{S_R} \sqrt{\frac{g}{2\equil{x}}} x
     + \frac{1}{S_R} u - \frac{\sqrt{2g\equil{x}}}{S_R} v
     + \frac{1}{S_R} \left( S_{S30} \sqrt{\frac{g\equil{x}}{2}}
     + \equil{u}
     - \sqrt{\frac{5g\equil{x}}{2}} \equil{v} \right)
    \text.
\end{align*}

\end{document}

